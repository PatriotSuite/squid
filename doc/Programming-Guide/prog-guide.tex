% $Id: prog-guide.tex,v 1.3 1997/01/15 05:16:37 wessels Exp $
\documentstyle[11pt,path,psfig]{report}
%\parskip		1ex
%\topmargin		0pt
%\headheight		0pt
%\headsep		0pt
%\marginparwidth	0pt
%\marginparsep		0pt
%\oddsidemargin		0pt
%\evensidemargin	0pt
%\textheight		9in
%\textwidth		6.5in
%\parindent		0em

\newenvironment{SS}{\singlespace}{}

\hyphenation{time-stamps time-stamp net-work ac-ces-ses ac-ces-sed}

\begin{document}
\bibliographystyle{ieeetr}

\author{Duane Wessels\\
Squid Developers}
\title{Squid Programmers Guide}

\maketitle

\begin{abstract}
Squid is a WWW Cache application developed by the National Laboratory
for Applied Network Research and members of the Web Caching community.
Squid is implemented as a single, non-blocking process based around
a BSD select() loop.  This document describes the operation of the Squid
source code and is intended to be used by others who wish to customize
or improve it.
\end{abstract}

%%
%% Chapter : INTRODUCTION
%%
\chapter{Introduction}

The Squid source code has evolved more from empirical observation and
tinkering, rather than a solid design process.  It carries a legacy of
being ``touched'' by numerous individuals, each with somewhat different
techniques and terminology.  

Squid is a single-process proxy server.  Every request is handled by
the main process, with the exception of FTP.  However, Squid does not
use a ``threads package'' such has Pthreads.  While this might be 
easier to code, it suffers from portability and performance problems.
Instead Squid maintains data structures and state information for
each active request.

The code is often difficult to follow because there are no explicit
state variables for the active requests.  Instead, thread execution
progresses as a sequence of ``handler functions'' which get called
when I/O is ready to occur.  As a handler completes, it will register
another handler for the next time I/O occurs.

Note there is only a pseudo-consistent naming scheme.  In most 
cases functions are named like {\tt moduleFooBar()}.  However, there
are also some functions named like {\tt module\_foo\_bar()}.

Note that the Squid source changes rapidly, and some parts of this
document may become out-of-date.  If you find any inconsistencies, please
feel free to notify us at {\tt squid-dev@nlanr.net}.

%%
%% Chapter : MAIN LOOP
%%
\chapter{The Main Loop: {\tt comm\_select()}}

At the core of Squid is the {\tt select(2)} system call.  Squid uses
{\tt select()} (or alternatively {\tt poll(2)} in recent versions) to 
process I/O on all open file descriptors.

\section{Comm Handlers}

For every open file descriptor, there are N types of handler functions.
\begin{itemize}
\item Read
\item Write
\item Timeout
\item Lifetime
\item Close
\end{itemize}

These handlers are stored in the {\em FD\_ENTRY} structure as defined in
\path|comm.h|.  {\tt fd\_table[]} is the global array of {\em FD\_ENTRY}
structures.  The handler functions are of type {\em PF}, which is a
typedef:
\begin{verbatim}
    typedef void (*PF) (int, void *);
\end{verbatim}
The close handler is really a linked list of handler functions.
Each handler also has an associated pointer ({\tt void *data)} to
some kind of data structure.

{\tt comm\_select()} is the function which issues the select() system
call.  It scans the entire {\tt fd\_table[]} array looking for handler
functions.  Each file descriptor with a read handler will be set in
the {\tt fd\_set} read bitmask.  Similarly, write handlers are scanned and
bits set for the write bitmask.  {\tt select()} is then called, and the
return read and write bitmasks are scanned for descriptors with pending
I/O.  For each ready descriptor, the handler is called.  Note that
the handler is cleared from the {\tt FD\_ENTRY} before it is called.

After each handler is called, {\tt comm\_select\_incoming()} is
called to process new HTTP and ICP requests.

Typical read handlers are
{\tt httpReadReply()},
{\tt diskHandleRead()},
{\tt icpHandleUdp()},
and {\tt ipcache\_dnsHandleRead()}.
Typical write handlers are
{\tt commHandleWrite()},
{\tt diskHandleWrite()},
and {\tt icpUdpReply()}.
The handler function is set with {\tt commSetSelect()}, with the
exception of the close handlers, which are set with {\tt
comm\_add\_close\_handler()}.

The close handlers are normally called from {\tt comm\_close()}.  
The job of the close handlers is to deallocate data structures 
associated with the file descriptor.  For this reason {\tt comm\_close()}
must normally be the last function in a sequence to prevent accessing
just-freed memory.

The timeout and lifetime handlers are called for file descriptors which
have been idle for too long.  They are futher discussed in a following 
chapter.

%%
%% Chapter : DATA STRUCTURES
%%
\chapter{Data Structures}
\section{Main Config}

%%
%% Chapter : STORAGE MANAGER
%%
\chapter{Storage Manager}

%%
%% Chapter : IP CACHE
%%
\chapter{IP Cache}

%%
%% Chapter : SERVER PROTOCOLS
%%
\chapter{Server Protocols}
\section{HTTP}
\section{FTP}
\section{Gopher}
\section{Wais}
\section{SSL}
\section{Passthrough}

%%
%% Chapter : TIMEOUTS
%%
\chapter{Timeouts}

%%
%% Chapter : EVENTS
%%
\chapter{Events}

%%
%% Chapter : ACCESS CONTROLS
%%
\chapter{Access Controls}

%%
%% Chapter : ICP
%%
\chapter{ICP}

%%
%% Chapter : CACHE MANAGER
%%
\chapter{Cache Manager}

%%%%%%
%%%%%% BIBLIOGRAPHY
%%%%%%
\newpage \bibliography{references}

% perl -ne 'printf ("\\nocite{\%s}\n", $1) if (/^@\w+{(\w+),/);' references.bib

\nocite{rfc850}
\nocite{rfc1123}

\end{document}
